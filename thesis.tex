%for a more compact document, add the option openany to avoid
%starting all chapters on odd numbered pages
\documentclass[12pt]{cmuthesis}

% This is a template for a CMU thesis.  It is 18 pages without any content :-)
% The source for this is pulled from a variety of sources and people.
% Here's a partial list of people who may or may have not contributed:
%
%        bnoble   = Brian Noble
%        caruana  = Rich Caruana
%        colohan  = Chris Colohan
%        jab      = Justin Boyan
%        josullvn = Joseph O'Sullivan
%        jrs      = Jonathan Shewchuk
%        kosak    = Corey Kosak
%        mjz      = Matt Zekauskas (mattz@cs)
%        pdinda   = Peter Dinda
%        pfr      = Patrick Riley
%        dkoes = David Koes (me)

% My main contribution is putting everything into a single class files and small
% template since I prefer this to some complicated sprawling directory tree with
% makefiles.

\definecolor{ValidRed}{rgb}{0.5,0,0}
\definecolor{TrueRed}{rgb}{1,0.6,0.2}
\definecolor{ValidBlue}{rgb}{0,0,0.5}
\definecolor{TrueBlue}{rgb}{0.2,0.6,1}

% some useful packages
\input{macros.tex}
%\draftstamp{\rm \today}{\rm DRAFT}

% \usepackage{titlesec,minitoc}  
% \titleclass{\part}{top}
% \titleformat{\part}
% {\centering\normalfont\Huge\bfseries} 
% {Foo} 
% {0pt} 
% {Bar}


\begin {document} 
\frontmatter

%initialize page style, so contents come out right (see bot) -mjz
\pagestyle{empty}

\title{ %% {\it \huge Thesis Proposal}\\
{\bf Robust Shape Optimization\\ using Differentiable Rendering}
%\\
}
\author{Sai Praveen Bangaru}
\date{\today}
\Year{2018}
\trnumber{revision 1.0.0}

\committee{
Ioannis Gkioulekas, Chair \\
Srinivasa Narasimhan \\
}



\disclaimer{Any opinions, findings, conclusions or recommendations
  expressed in this publication are those of the author and do not
  necessarily reflect the views of any sponsoring institution or
  government.}

% copyright notice generated automatically from Year and author.
% permission added if \permission{} given.

\keywords{logical frameworks, linear logic, ordered logic, operational
  semantics}

\maketitle

% XXX MAKE DEDICATION
%\begin{dedication}
%To my grandparents
%\end{dedication}

\pagestyle{plain} % for toc, was empty

%% Obviously, it's probably a good idea to break the various sections of your thesis
%% into different files and input them into this file...

\begin{abstract}
  Most radiometric methods for the extraction of shape suffer from various approximations and assumptions about light transport, that ultimately make them unsuitable for certain classes of objects. Some of the most common assumptions include lambertian reflectance, absence of interreflections and distant light sources. Our goal is to construct a light-transport based framework that can be used to retrieve shape as well as reflectance properties without any of the above assumptions. The key component in this framework is a versatile differentiable path tracer which can not only compute the differentials of global BRDF parameters w.r.t image intensity, but also the differentials of each individual normal of the object w.r.t its image intensity. We further show how this can be used as a component in a non-linear optimization problem that minimizes the error between simulated and actual images. Finally, this thesis presents reconstruction results from both simulated as well as real-world datasets.
\end{abstract}

% XXX MAKE ACKNOLWEDGEMENTS
\begin{acknowledgments}
 
\end{acknowledgments}

\tableofcontents
\listoffigures % XXX Do I want this?
% \listoftables XXX Probably don't need this at all.

\mainmatter

%% Double space document for easy review:
%\renewcommand{\baselinestretch}{1.66}\normalsize

% The other requirements Catherine has:
%
%  - avoid large margins.  She wants the thesis to use fewer pages, 
%    especially if it requires colour printing.
%
%  - The thesis should be formatted for double-sided printing.  This
%    means that all chapters, acknowledgements, table of contents, etc.
%    should start on odd numbered (right facing) pages.
%
%  - You need to use the department standard tech report title page.  I
%    have tried to ensure that the title page here conforms to this
%    standard.
%
%  - Use a nice serif font, such as Times Roman.  Sans serif looks bad.
%
% Other than that, just make it look good...


% Introduction
\chapter{Introduction}
\label{chapter-introduction}

% The topic of this thesis is the specification of {\it evolving systems}

% The {\it lingua franca} of research in programming languages and
% logics is the {\it inductive definition}. Type systems are defined in
% terms of inductive definitions like $\Gamma \vdash e : \tau$
% (within the typing context $\Gamma$, the term $e$ has type $\tau$).
% % \[
% % \infer
% % {\Gamma \vdash x : \tau \mathstrut}
% % {x{:}\tau \in \Gamma \mathstrut}
% % \quad
% % \infer
% % {\Gamma \vdash {\sf z} : {\sf nat} \mathstrut}
% % {}
% % \quad
% % \infer
% % {\Gamma \vdash {\sf s}\,e : {\sf nat} \mathstrut}
% % {\Gamma \vdash e_1 : {\sf nat}
% %  &
% %  \Gamma \vdash e_2 : {\sf nat}}
% % \]
% Operational semantics are also defined in terms of inductive
% definitions: a {\it small-step} semantics has the form $e \mapsto e'$
% (the expression or machine state $e$ can transition to $e'$), and a
% {\it big-step} semantics has the form $e \Downarrow v$ (the expression
% or machine state $e$ can ultimately produce the terminal machine state
% $v$). 

% Proof assistants -- computer programs that help programming language
% researchers specify systems, explore their behavior, and prove
% properties of their behavior -- obviously must therefore be able to
% talk about inductive definitions. The most common way proof assistants
% do this is by directly incorporating a notion of inductive definition
% into their framework. Coq \cite{}, Agda \cite{}, ATS \cite{},
% Isabelle/HOL \cite{}, Matita \cite{}, and Abella \cite{} all work this
% way; inductive types are introduced

%  which covers all the theorem provers used to prove the POPLMark
% challenge except for Twelf.

% used in these domains therefore universally
% include a notion of {\it inductive definition} as the primary form of

\section{Overview}
{\it Photometric stereo}, the de-facto term used for extracting shape from multiple directionally lit images, is almost as old as computer vision itself. At it's core, it is fundamentally the solution to a linear system that relies on the fact that exitant radiance from a lambertian surface patch is proportional to the dot product of incident light direction and the path normal. When it came out, this algorithm was extremely powerful. For most objects, it could estimate a per-pixel normal, leading to very accurate shape reconstructions.\\
It does, however, make several assumptions:
\begin{enumerate}
    \item Distant lights (Directional lighting)
    \item Convex shape (No multiple bounces/interreflections)
    \item Lambertian BSDF ($\mathbf{n}.\mathbf{l}$ shading)
\end{enumerate}

These assumptions do not sit very well with real world objects. For instance, the lambertian BSDF assumption breaks down for even the most diffuse objects, because of a grazing-angle \textit{Fresnel} component. This is the same phenomenon where seemingly rough surfaces start to behave like mirrors when viewed at an angle close to parallel with the surface.

The assumption of convexity also does not hold true for most objects, though its effects are harder to pin down accurately. Interreflections occur whenever two patches of an object face each other ($\mathbf{n_1}.\mathbf{n_2} > 0$. This means they are most pronounced near sharp inward edges. There also extremely concave surfaces like bowls and mugs that are rarely, if ever, used in photometric stereo papers because of extreme inter-reflections.

The final assumption, directional lighting, is something that this thesis does not explicitly tackle, but will demonstrate that the framework can be trivially extended to handle any form of lighting. Most photometric stereo methods are limited by assumption that light hits every patch at the same angle. This simplifies the optimization problem significantly since the solution is independent of the relative distance between the light and the mesh. (TODO: Near-light photometric stereo)

In practice, these assumptions can be ignored in a some cases, since grazing angle reflections don't account for much of the image, and based on the shape, interreflections may only contribute near sharp edges. In this thesis, we look at objects where we cannot ignore these assumptions. This includes either specular surfaces or highly concave objects, or both.

\section{Background}
The following section contains some background information that are relevant to the components of the framework presented in this thesis.
\subsection{Monte Carlo path tracing}
At the heart of any analysis-by-synthesis problem is the \textit{synthesis}. Since our framework aims to account for as many light transport phenomena as possible, it makes sense to use a fully-fledged phyically-based rendering method, like path tracing. \\
Path tracing is an umbrella term for a family of Monte Carlo estimators that are used to sample paths of light through a scene. One could fill a book with the various sampling techniques, each of which better than the last. However, they all estimate the same fundamental recursive equation known as the \textit{rendering equation}, one of the keystones of computer graphics.
(TODO: rendering equation)

For the purpose of this thesis, we use the popular algorithm BDPT, which represents a sweet spot in the trade-off between complexity and sample efficiency. 
At each iteration, the BDPT algorithm randomly sample two paths $\mathcal{L}$ and $\mathcal{C}$ from the light source and camera respectively of lengths $S$ and $T$ (which are determined by a Russian roulette factor). By connecting every point in $\mathcal{L}$ to every point in $\mathcal{C}$, BDPT creates a set of paths, along with the probability of path $s,t$ denoted by $p(\{\mathcal{L}_s,\mathcal{C}_t\})$.
The final intensity is then estimated by the equation:
(TODO: Place path integral here).

Note that for our framework, we only require that the integral be written as an integral over a set of sampled paths and their probabilities. This means we do not necessarily require BDPT. Any similar algorithm, like MLT or PSSMLT would also work once they generate a set of paths.

\subsection{Surface integration}
Surface integration is an key part of our framework, but given the extensive literature on the topic of gradients-to-depth algorithms, there is no need to come up with our own. Since we work with fairly noisy normals (an unfortunate side effect of stochastic gradient descent algorithms), a weighted integration algorithm like Poisson surface reconstruction is a perfect fit for our framework.

Weighted Poisson surface reconstruction can plainly expressed as the linear least squares solution to the following set of constraints.
(TODO: Add constraints)

\subsection{Adaptive Gradient Descent}
Gradient descent is a fundamental part of the non-linear optimizaton problem that is responsible for inferring the normals as well as BSDF parameters.
(TODO: Add SGD rules)

The need for Adaptive gradient descent algorithms stems from the fact that some parameters of a model can have a more drastic effect on the error than others, while also operating on the same scale (and range of values). This discrepancy introduces a catch-22 situation where a learning rate that is too small will take forever to converge and a learning rate that is too large will cause some of the parameters to diverge quickly (or behave unpredictably). This necessitates adaptive gradients, which take the form of many popular algorithms like AdaGrad, AdaDelta, Adam and RMSProp. We choose Adam since it is also the algorithm of choice for many neural networks, and has provably good characteristics.

(TODO: Add Adam rules)

\part{Focusing substructural logics}

% Linear logic
\input{chapter-foc}

% Substructural logic
\input{chapter-order}

% Substructural logical specifications
\chapter{Differentiable Rendering}
\label{chapter-framework}


\section{Estimating Differentials}
% Big section. Main part of thesis.


\section{Practical differentiable rendering}
In this section, we discuss some of the practical considerations of the algorithm provided above.
These (fairly straightforward) modifications allow for fast and hassle-free optimization that might otherwiae be subject to numerical issues.
\subsection{Spherical projection}
Estimate projected gradients to speed up the 

\subsection{Pre-multiplication}
Multiply samples of the gradient tensor with the error image to form a sample of the gradient vector. This minimizes memory usage considerably since we no longer need storage for (WxHxN) elements.

\subsection{Set-Adaptive gradient descent}


\part{Substructural operational semantics}

% On logical correspondence
\input{chapter-correspondence}

% Ordered abstract machines
\input{chapter-absmachine}

% Destination-passing style
\input{chapter-destinations}

% Linear logical approximation
\input{chapter-approx}

\part{Reasoning about substructural logical specifications}

\input{chapter-gen}

\input{chapter-safety}

\chapter{Conclusion}
\label{chapter-conclusion}

\section{Future Directions}
\subsection{Adaptive sampling}
One interesting technique that can be used to great effect is adaptive sampling of light paths. Instead of using a fixed number of paths per pixel, it makes sense to prioritize the pixels with higher error, since they have a greater contribution to the total error. This is a form on importance sampling applied to differentiable rendering, and will serve top significantly speed up the optimization process, allowing for greater fidelity within the same time bounds.

\appendix

% \chapter{Programming with canonical forms}

% \newcommand{\F}[1]{\ensuremath{F({#1})}}
% \newcommand{\G}[1]{\ensuremath{G(\textcolor{TrueBlue}{#1})}}
% \newcommand{\upX}[1]{\ensuremath{{\uparrow}\textcolor{ValidBlue}{#1}}}
% \newcommand{\downX}[1]{\ensuremath{{\downarrow}\textcolor{ValidRed}{#1}}}
% \newcommand{\upA}[1]{\ensuremath{{\uparrow}\textcolor{TrueBlue}{#1}}}
% \newcommand{\downA}[1]{\ensuremath{{\downarrow}\textcolor{TrueRed}{#1}}}

% \newcommand{\valid}[1]{\ensuremath{{\downarrow}\textcolor{ValidBlue}{{#1}\,\mathit{valid}}}}
% \newcommand{\true}[1]{\ensuremath{{\downarrow}\textcolor{TrueBlue}{{#1}\,\mathit{true}}}}

% \newcommand{\ajseq}[2]{\ensuremath{\mathstrut{#1} \vdash {#2}}}
% \newcommand{\ajinv}[3]{\ajseq{{#1}; {#2}}{\textcolor{ValidRed}{#3}}}
% \newcommand{\ajrfoc}[2]{\ajseq{{#1}}{[\textcolor{ValidBlue}{#2}]}}
% \newcommand{\ajlfoc}[3]{\ajseq{{#1} [{#2}]}{\textcolor{ValidRed}{#3}}}
% \newcommand{\ajAseq}[3]{\ensuremath{\mathstrut{#1} \vdash {#2}}}
% \newcommand{\ajAinv}[4]{\ajseq{{#1}; {#2}}{\textcolor{ValidRed}{#3}}}
% \newcommand{\ajArfoc}[3]{\ajseq{{#1}}{[\textcolor{ValidBlue}{#2}]}}
% \newcommand{\ajAlfoc}[4]{\ajseq{{#1} [{#2}]}{\textcolor{ValidRed}{#3}}}
% \newcommand{\ajXseq}[2]{\ensuremath{\mathstrut{#1} \vdash {#2}}}
% \newcommand{\ajXinv}[3]{\ajseq{{#1}; {#2}}{\textcolor{ValidRed}{#3}}}
% \newcommand{\ajXrfoc}[2]{\ajseq{{#1}}{[\textcolor{ValidBlue}{#2}]}}
% \newcommand{\ajXlfoc}[3]{\ajseq{{#1} [{#2}]}{\textcolor{ValidRed}{#3}}}

% \begin{figure}
% \fbox{\ajXrfoc{\Gamma}{X^+}}
% \[
% \infer[x^+_R]
% {\ajrfoc{\valid{x^+}}{x^+}}
% {}
% \qquad
% \infer[{\downarrow}_{XR}]
% {\ajXrfoc{\Gamma}{\downX{X^-}}}
% {\ajXinv{\Gamma}{\cdot}{X^-}}
% \qquad
% \infer[G_R]
% {\ajXrfoc{\Gamma}{\G{A}}}
% {\ajArfoc{\Gamma}{\cdot}{A}}
% \]

% \fbox{\ajArfoc{\Gamma}{\Delta}{A^+}}
% \[
% \infer[a^+_R]
% {\ajArfoc{\cdot}{a^+}{a^+}}
% {}
% \qquad
% \infer[{\downarrow}_{AR}]
% {\ajArfoc{\Gamma}{\Delta}{\downA{A^-}}}
% {\ajAinv{\Gamma}{\Delta}{\cdot}{A^-}}
% \]

% \caption{Focused adjoint logic}
% \end{figure}


%\include{appendix}

\backmatter

%\renewcommand{\baselinestretch}{1.0}\normalsize

% By default \bibsection is \chapter*, but we really want this to show
% up in the table of contents and pdf bookmarks.
\renewcommand{\bibsection}{\chapter{\bibname}}
%\newcommand{\bibpreamble}{This text goes between the ``Bibliography''
%  header and the actual list of references}
\bibliographystyle{alphaurl}
\bibliography{ref} %your bib file

\end{document}
